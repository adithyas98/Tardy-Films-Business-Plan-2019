\documentclass[12pt]{article}
\usepackage[utf8]{inputenc}
\usepackage[margin=1in]{geometry}
\usepackage{setspace}

\title{Tardy Films Business Plan}
\date{April 2019}
\author{Adi Shastry, Trisha Ramdhoni, and Fatimah Ali\\ adi.shastry@gmail.com}

\begin{document}

\maketitle
\newpage
\tableofcontents
\newpage
\doublespacing
\section{Executive Summary}
Tardy Films is revolutionizing the business of selling movie tickets to a waning customer base. The service will release tickets at significantly marked down prices at the optimal time in an effort to generate more ticket sales on the theater's behalf. Given the fixed cost nature of running a movie theater, this sort of incremental revenue and profit will significantly increase the revenue of theaters. Our business growth is driven on three major factors: acquiring users, signing on more theaters, and finally selling more tickets to the existing user base. 

\newpage
\section{Company Overview}
Tardy Films is still in the starting process and has yet to be incorporated. When this stage is reached, the company will be incorporated in Delaware as an LLC because of the legal and operational benefits that come with this legal status. Our Company will be headquartered in Boston which will be the initial geographical location the company will operate in. 

\section{Industry Analysis}
The movie theater industry is in the maturity or decline phase of the industry growth model. This can be seen by the steady decline in the number of movies that people are attending on an annual basis. This steady decline comes as a result of the advancement of streaming technology pioneered by Netflix and others like Amazon Prime, or Hulu. The convenience of these services makes it a very easy decision for most consumers to simply stay in the comfort of their homes instead of going to the theater for the newest flick. 

As a result of this recent trend, many movie theaters are looking to add more value to their service by creating dine-in movies and developing better technology in an effort to make a more immersive experience. These strategies have worked only to a limited degree, essentially increasing revenues for a short period of time, and ultimately not affecting the overall profit structure of the theater.This proves that more drastic changes need to be implemented in order to continue to remain profitable in the theater industry.

One such solution was Movie Pass. Movie Pass offered a subscription model similar to Netflix, but allowed their subscribers to go to movies theaters as frequently as once per day. This was a huge success in terms of the number of subscribers that the company acquired in such a short period of time. The problem with Movie Pass, however, was their underlying business model. The model was created under the same principles of the insurance industry, but in the case of the theater industry it proved to not actually work. In other words, the company was built on the premise that most subscribers would not go to movies often, allowing Movie Pass to distribute the subscription revenue over the cost. However, this case study gives some key insights about the behavior of consumers of the movie theater industry: the idea of a discounted ticket is one that will get people to leave their homes to go to the movies.

\section{Customer Analysis}
As mentioned in the previous section, the primary customers of the movie theater industry seem to be shifting towards the convenience of streaming services like Netflix. The total uptick in revenues for theater companies like AMC have not increased due to their investments in creating better experiences for their customers. We have seen through the case of Movie Pass that a per ticket price reduction is actually very appealing to consumers of movies. It is also clear that the management at these various businesses are reluctant to offer any price reductions. Tardy Films will provide the benefit of selling tickets at full price while also providing the incremental revenue gained through a price reduction. For example, the opening three weeks of any movie would be unaffected by a partnership with Tardy Films - giving the benefit of full price tickets. In this scenario, Tardy Films would start to adjust pricing by offering discounts on showing beyond the first three weeks. Note, however, that the timeline will be subject to change based on the predictions of the AI we are implementing and the specifications of the theater we have partnered with.

\section{Competitive Analysis}
While Tardy Films does not have any direct competitors, it has numerous possible substitutes such as Netflix, Hulu, Amazon Prime, AMC Stubs Alist, and Cinemark's Movie Club. It is clear from the Movie Pass case that customers would go to the movies if there was a significant price reduction in movie tickets. This shows that the main issue that theaters are facing is a problem with accurately predicting demand for movies and pricing accordingly. This would in turn bring more customers to the theater and, more importantly, get more customers buying the concessions that the theater sells.
\section{Operations}
The operations of Tardy Films will take three steps: 
\begin{enumerate}
	\item Acquiring users, theaters, and theater sales data
	\item Packaging and releasing deals at the appropriate time using Artificial Intelligence
	\item Notifying users and make the ticket sales
\end{enumerate}
In step one, we will need to create the market place between theaters and customers. This step also involves inputting accurate sales data into the machine learning model used in step two to accurately teach the model when to release tickets. Step two will correctly price the tickets based on the theater's request and release them at the appropriate time. Finally, in step three, users will be able to buy the tickets and will receive a voucher that will let them into the movie.
\section{Financial Model}
Tardy Films will have three main revenue streams originating from the theaters, users, and finally production houses. The money from the theaters will be as a result of a subscription payment that we require for the theater to be on the platform. This ensures that Tardy Films is compensated from day one of signing on a new theater (supplier). The next source of revenue will be from the actual sale of tickets on our platform. This will ensure to our suppliers that it is in our best interest to actually follow through with our promises of providing ticket sales. Finally, we will be able to sell our insights derived from the transactions occurring on the site to various production houses in an effort to help them produce movies Americans want to see.

Through more granular analysis of the overall financial model of Tardy Films, it is clear that the company can grow in three major ways: acquiring more users, acquiring more theaters, and finally selling more tickets to the current user base. The first and second go hand in hand because of the leverage that each would give over the other. In other words, the more users we have the easier it will be to sign on new theaters and vice versa. Given this model for theater and user acquisition, the act of getting our initial users and theaters on the platform will have to be done with extreme care. Finally, selling more to the existing user base can only be achieved by improving our product in an effort to cater perfectly to the demands of each and every user both on a micro scale and on a macro scale.

The following table shows the highlights of our projections for the upcoming five years.

\begin{table}[h!]
\begin{tabular}{|l|r|r|r|r|r|}
\hline
\textbf{5 Year Projections} & \multicolumn{1}{l|}{}     & \multicolumn{1}{l|}{}     & \multicolumn{1}{l|}{}     & \multicolumn{1}{l|}{}     & \multicolumn{1}{l|}{}     \\ \hline
\textbf{Year}               & \multicolumn{1}{l|}{2019} & \multicolumn{1}{l|}{2020} & \multicolumn{1}{l|}{2021} & \multicolumn{1}{l|}{2022} & \multicolumn{1}{l|}{2023} \\ \hline
                            & \multicolumn{1}{l|}{}     & \multicolumn{1}{l|}{}     & \multicolumn{1}{l|}{}     & \multicolumn{1}{l|}{}     & \multicolumn{1}{l|}{}     \\ \hline
\textbf{Assumptions}        & \multicolumn{1}{l|}{}     & \multicolumn{1}{l|}{}     & \multicolumn{1}{l|}{}     & \multicolumn{1}{l|}{}     & \multicolumn{1}{l|}{}     \\ \hline
Revenue Growth Rate         &                           & 200\%                     & 200\%                     & 200\%                     & 150\%                     \\ \hline
Gross Profit Margin Rate    & 93.3\%                    & 93.0\%                    & 93.0\%                    & 93.0\%                    & 93.0\%                    \\ \hline
Profit Margin               &                           & 50.0\%                    & 50.0\%                    & 50.0\%                    & 50.0\%                    \\ \hline
                            & \multicolumn{1}{l|}{}     & \multicolumn{1}{l|}{}     & \multicolumn{1}{l|}{}     & \multicolumn{1}{l|}{}     & \multicolumn{1}{l|}{}     \\ \hline
\textbf{Revenue}            & \$1,870.00                & \$5,610.00                & \$16,830.00               & \$50,490.00               & \$126,225.00              \\ \hline
\textbf{Gross Profit}       & \$1,745.39                & \$5,217.30                & \$15,651.90               & \$46,955.70               & \$117,389.25              \\ \hline
\textbf{Net Income}         & (\$9,868.25)              & \$2,608.65                & \$7,825.95                & \$23,477.85               & \$58,694.63               \\ \hline
\end{tabular}
\end{table}

A more detailed version of our financial model can be seen in the attached pages.



















\end{document}
